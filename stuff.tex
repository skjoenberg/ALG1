\documentclass{article}
\usepackage[utf8]{inputenc}
\usepackage[T1]{fontenc}
\usepackage{stmaryrd}
\usepackage{fancyhdr}
\usepackage{natbib}
\usepackage{graphicx}
\usepackage{amsmath}
\usepackage{amssymb}
\usepackage{pgf}
\usepackage{enumerate}
\usepackage{tikz}
\usetikzlibrary{arrows,automata}

\begin{document}

\section*{Grupper}
\textbf{def.}\\
Hvis $G$ er en gruppe under operationen $\cdot$, så gælder
\begin{enumerate}
\item Identiten af $G$ er unik
\item $\forall{a} \in G, a^{-1}$ er entydigt bestemt
\item $\{ \forall{a} \in G | (a^{-1})^{-1} = a\}$
\item $(a \cdot b)^{-1} = (b^{-1}) \cdot (a^{-1})$
\item Assosiativ
\end{enumerate}

\section*{Bijektiv, surjektiv}
\subsection*{Bijektion}
"A bijection (or bijective function or one-to-one correspondence) is a function between the elements of two sets, where every element of one set is paired with exactly one element of the other set, and every element of the other set is paired with exactly one element of the first set."\\
Wiki.
\subsection*{Bijektion}

\section*{Permutationer}
\textbf{def.}\\
En permutation af et sæt $A$ er en bijektion fra $A$ til sig selv.\\
$S_n$ er en symmetrisk gruppe af grad $n$.

\section*{Legemer}


\section*{Undergrupper}
\textbf{def.}\\
$G$ gruppe. En delmængde $H$ af $G$ kaldes en undergruppe af $G$ skrives $h \leq G$, hvis $H \neq \emptyset$ og $H$ stabil under komposition og inversdannelse.
\begin{enumerate}
  \item $\forall{x, y} \in H : xg \in H$
  \item $\forall{x} \in H : x^{-1} \in H$
\end{enumerate}
Hvis $H \leq G$ da er $H$ selv gruppe med komposition fra $G$. Da $H \neq \emptyset$, findes $x \in H$.
\begin{enumerate}
  \item $x^{-1} \in H$
  \item $1 = x x^{-1} \in H$
\end{enumerate}
\textbf{prop.}\\
$G$ er en gruppe. $H \subset G$. Da er $H \leq G$ hvis og kun hvis:
\begin{enumerate}
  \item $H \neq \emptyset$
  \item $\forall{x, y} \in H : x y^{-1} \in H$
\end{enumerate}
Hvis $H \neq \emptyset$ er endelig, da er det tilstrækkeligt at $H$ er stabil under komposition.

\textbf{Sætning} \textit{Lagranges}\\
$G$ endelig, $G \leq G$. Da er $|H| | |G|$, og $\frac{|G|}{|H|}$ er antallet af sideklasser af $H$ i $G$.

\textbf{Def.}\\
$G$ gruppe, $H \leq G$. Index af $H$ i $G$ $[G:H]$ antal venstresideklasser af $H$ i $G$.\\
Hvis $G$ er endelig: $|G| = |H| [G:H]$ \textit{(Lagranges)}.

\textbf{Sætning} \textit{Cancley}\\
$G$ endelige gruppe, p primtal med $p \mid |G|$.\\
Da har $G$ element af orden $p$ (og dermed også en undergruppe af orden $p$).

\textbf{Sætning} \textit{Sylow}\\
$G$ endelig gruppe, $p$ primtal, $|G| = P^a \cdot m$, $p \nmid m$.\\
Da findes undergruppe af orden $p^a$ (såkaldt $p$-Sylowgruppe).\\

\textbf{Prop.}\\
$G$ endelig, $H, K \leq G$. Da er $|HK| = \frac{|H||K|}{|H \cap K|}$.

\section*{Normale undergrupper}
\textbf{Def.}\\
G gruppe, $H \leq G$, venstre- og højresideklasse af $H$ i $G$:\\
For $g \in G$\\
$gH := \{ gh \in H\}$\\
$Hg := \{ hg \in H\}$\\
\textbf{Def.}\\
$N \leq G$ kaldes normal, skrives $N \trianglelefteq G$,\\
hvis : $\forall{g \in G} : gNg^{-1} = \{ gng^{-1} | n \in N\} = N$\\
\textbf{Prop.}\\
Lad $n \leq G$. Da er følgende betingelser ækvivalente:
\begin{enumerate}
  \item $N \trianglelefteq G$
  \item $N_G(N) = G$
  \item $\forall{g \in G} : gN = Ng$
  \item $\forall{g \in G} : gNg^{-1} \subset N$
\end{enumerate}

\textbf{Sætning}\\
$G$ gruppe, $N \trianglelefteq G$. Da er afb. :
$\phi : G \to G/N$, givet ved\\
$\phi(g) := gN \in G/N$ er injektiv homomorfi med kerne $N$.\\
$\phi :$ den kanoniske homorfi (af $G$ på $G/N$).
Bog kalder det $Natural projection$.\\

\section*{Kvotient grupper}

\section*{Kerner}
$C_G$
\section*{Centre}

\begin{align*}
  Z(G) &:= \{ g \in G | \forall h \in G : g h g^{-1} = h \}\\
  &= \{ g \in G | \forall h \in G : gh = hg \} \leq G
\end{align*}
\section*{Stabilisator}
\textbf{def.}\\
Hvis $A \subset G$ og $g \in G$, defineres:\\
$$g A g^{-1} = \{ g a g^{-1} | a \in A\}$$
konjugeret med A.
\section*{Normalisatoren}
\textbf{def.}\\
$G$ gruppe, $A \subset G$. Normalisatoren af $A$ er:\\
$$N_G(A) = \{ g \in G | g A g^{-1} = A\} \leq G$$

\section*{Cykliske grupper}
\textbf{def.}\\
Gruppe $G$ kaldes cyklisk, hvis der findes
$x \in G$, så $G = \{ x^n | n \in \mathbb{Z} \}$

\section*{Sideklasser}
Engelsk: \textit{coset}.\\

\section*{Isomorfi}
1. isomorfi sætning\\
$G, H$ er grupper, $\phi : G \to H$ er en homomorfi.
\begin{enumerate}
  \item $Ker(\phi) \triangleleft G$
  \item $Im(\phi) \leq H$
  \item $Im(\phi) = \phi(G) \cong G / Ker(\phi)$
\end{enumerate}

2. isomorfi sætning\\
$A, B$ er grupper. $A, B \neq G$. $A \neq N_G(B)$. Da er
\begin{enumerate}
  \item $AB \leq G$
  \item $B \triangleleft AB$
  \item $A \cap B \triangleleft A$
  \item $A \mid (A \cap B) \cong AB/B$
\end{enumerate}

3. isomorfi sætning\\

4. isomorfi sætning.\\
$G$ gruppe, $N \triangleleft G$. $\overline{G} := G / N$. $\pi : G \to G / N$ kanonisk homomorfi. $\pi(g) := g N$ (Natural projection).\\
For $U \leq G$ defineres $\overline{U} := \pi(U)$ er en undergruppe af $\overline{G} = G / N$.\\
For $V \leq G$ $\pi^{-1}(v) := \{ g \in G | \pi(g) \in V\}$. Urbilledet af $V$ under $\pi$. Da er $\pi^{-1}(V)$ er en undergruppe af $G$ indeholdende $N$.

\section*{Endnu ikke placeret}
\begin{enumerate}
  \item $A \leq B \iff \overline{A} \leq \overline{B}$
  \item $A \leq B \Rightarrow [B:A] = [\overline{B}:\overline{A}]$
  \item $\overline{<A, B>} = <\overline{A}, \overline{b}>$
  \item $\overline{A \cap B} = \overline{A} \cap \overline{B}$
  \item $A \triangleleft G \iff \overline{A} \triangleleft \overline{G} = G / N$
\end{enumerate}

\end{document}