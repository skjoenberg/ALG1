\documentclass[11pt]{article}
\usepackage[a4paper, hmargin={2.8cm, 2.8cm}, vmargin={2.5cm, 2.5cm}]{geometry}
\usepackage{eso-pic} % \AddToShipoutPicture
\usepackage{graphicx}
\usepackage[utf8]{inputenc}
\usepackage[T1]{fontenc}
\usepackage{stmaryrd}
\usepackage{natbib}
\usepackage{float}
\usepackage{amsmath}
\usepackage{amssymb}
\usepackage{dsfont}
\usepackage{pgf}
\usepackage{tikz}
\usepackage{fancyhdr}
\usepackage{lastpage}
\usepackage{booktabs}
\usepackage{lmodern}
\usepackage[normalem]{ulem}
\usetikzlibrary{arrows,automata}
\setlength{\parindent}{0in}

\pagestyle{fancy}
\fancyhf{}
\setlength{\headheight}{2.5em}
\renewcommand{\headrulewidth}{0pt}
\fancyhead[R]{Sebastian Andersen Algebra 1 (2015)}
\fancyfoot[C]{page \thepage\ of \pageref{LastPage}}
\begin{document}

\section*{Grupper}
\subsection*{Definition}
\begin{enumerate}
  \item Lukket under komposition (gruppeoperationen).
  \item Assosiativ.
  \item Unikt identitetselement.
  \item Lukket under inversdannelse og hver invers er unik.
\end{enumerate}

En gruppe er abelsk, hvis alle elementer kommuterer.
\subsection*{Orden}
Ordenen af en gruppe $G$ er givet ved antallet af elementer i gruppen.\\
Ordenen af et element $x \in G$ er givet ved det laveste heltal $n$, så $x^n = 1$ (hvor $1$ er identitetselementet).\\
Et element har uendelig orden, hvis der ikke findes et $n$, som angivet ovenfor.\\
Et element har orden $1$, hvis og kun hvis det er identitetselementet.
\subsection*{Cancellation laws}
\begin{align*}
  au &= av, \text{ så gælder } u = v\\
  ub &= vb, \text{ så gælder } u = v
\end{align*}
\subsection*{Generatorer}
Generatoren for en gruppe er en delmængde, givet ved, at alle elementer i gruppen kan blive udtrykt ved en endelig kombination (under gruppeoperationen) af elementerne (og deres inverse) i delmængden.
\subsection*{Diedergruppen af orden 2n}
Diedergruppen af orden $2n$ skrives $D_{2n}$.\\
For alle $n \geq 3$, lad $D_{2n}$ være sættet af symmetrier af en $n$-gon.\\
Der findes to basale operationer på diedergruppen. Henholdsvis spejlinger $s$ og rotationer $r$.\\
Alle elementer i diedergruppen kan beskrives ved $s^k r^i$, hvor $k = 0$ eller $1$ og $0 \leq i \leq n-1$.\\
Ordenen af diedergruppen er $|D_{2n}| = 2n$.\\
$r s = s r^{-1}$.\\
$r^i s = s r^{-i}$.\\
Diedergruppen kan også beskrives ved generatoren: $D_{2n} = \langle r, s | r^n = s^2 = 1, rs = s r^{-1} \rangle$.
\subsection*{Symmetriske grupper}
Lad $\Omega$ være en ikke-tom mængde og lad $S_{\Omega}$ være en mængde af alle bijektioner fra $\Omega$ til sig selv. $S_{\Omega}$ kaldes den symmetriske gruppe af mængden $\Omega$. $S_{\Omega}$ indeholder alle permutationer af $\Omega$.\\
I specialtilfældet $\Omega = \{ 1,2,3, \hdots, n \}$ skrives den symmetriske gruppe $S_n$.\\
Ordenen af $|S_n| = n!$.\\
En cykel svarer til en lige permutation, hvis den har et ulige antal (og omvendt).

\subsection*{Den alternerende gruppe}
\textit{Den alternerende gruppe} er en gruppe af alle lige permutationer af et endeligt sæt. Den alternerende gruppe af $\{ 1,2,3 \hdots, n \}$ kaldes den alternerende gruppe af grad $n$ (b$A_n$).\\
Den alternerende gruppe har orden $|A_n| = \frac{n!}{2}$.
\subsection*{Kleins 4-gruppe}
\subsection*{Matrixgrupper}
\subsection*{Kvaterniongruppen Q8}
2.5
\subsection*{Homomorphier}
\textbf{Definition.}\\
Lad $G$ og $G$ være grupper. En afbildning $\phi : G \to H$ sådan at
$$\phi(xy) = \phi(x) \phi(y), \forall{x,y} \in G$$
er en homomorfi.

\subsection*{Isomorphier}
\textbf{Definition.}\\
Afbildningen $\phi : G \to H$ er en isomorfi, hvis følgende er opfyldt:
\begin{enumerate}
  \item $\phi$ er en homomorfi.
%  \item $|G| = |H|$ (lige mange elemter).\\
  \item $\phi$ er bijektive (injektiv og surjektiv).
\end{enumerate}

Hvis der findes en isomorfi mellem $G$ og $H$, siges det, at $G$ er isomorf med $H$ ($G \simeq H$).\\
$$\phi \text{ er injektiv } \iff Ker(\phi) = \{e\}$$
Injektivitet kan også vises ved $\phi(x) = \phi(y) \iff x = y$.

$$G / Ker(\phi) = H$$

\subsection*{Gruppevirkninger}
\textbf{Definition.}\\
En gruppevirkning af en gruppe $G$ på en mængde $A$ er givet ved en afbildning $G \times A \to A$, der opfylder følgende:\\
\begin{enumerate}
  \item $\pi(g_1) \cdot \pi(g_2) \cdot a = \pi(g_1 g_2) \cdot a$
  \item $\pi(1) \cdot a = a$.
\end{enumerate}
En gruppevirkning er altså, at for hvert element $g \in G$ fåes en permutation af mængden $A$.\\
En gruppevirkning kan også ses som en homomorfi $\pi : G \to bij(x,x)$.\\
En gruppevirkning er en ækvivalens relation på $A$.\\
$$x \sim y \iff \exists{g \in G} \text{ sådan at } \pi(g)x = y$$

\section*{Undergrupper}

\subsection*{Definition}
Lad $H$ og $G$ være grupper. $H$ er en undergruppe af $G$, skrives$H \leq G$, hvis $H$ er en ikke-tom delmængde af $G$. $H$ er lukket under komposition og inversdannelse.

\subsection*{Undergruppekriteriet}
$H$ er en undergruppe af $g$ hvis og kun hvis:
\begin{enumerate}
  \item $H \neq \emptyset$
  \item for alle $x, y \in H, x y^{-1} \in H$
\end{enumerate}

\subsection*{Centralisator}
\textbf{Definition}:\\
$$C_G(A) = \{ g \in G \mid g a g^{-1} = a,  \forall{a} \in A \}$$
Altså er centralisatoren en delmængde af $G$, der kommuterer med alle elementer i $A$.\\

\subsection*{Normalisator}
\textbf{Definition}:\\
$$N_G(A) = \{ g \in G \mid gAg^{-1} = A\}$$

Altså er normalisatoren en delmængde af $G$, der laver en afbildning af $A$ over i sig selv.\\

Den primære forskel på centralisator og normalisator er, at centralisatoren sender et element over i sig selv, hvorimod normalisatoren sender elementet over i mængden $A$.

\subsection*{Kernen}
Lad $\phi$ være en homomorfi $\phi : G \to H$. Kernen af $\phi$ er mængden

$$Ker(\phi) = \{ g \in G \mid \phi(G) = 1 \}$$

\subsection*{Centrum}

\subsection*{Stabilisatorer og kerner af gruppevirkninger}

\subsection*{Undergruppe frembragt af en delmængde}

\subsection*{Baner}
Da en gruppevirkning er en ækvivalensrelation, kalder vi dens ækvivalensklasser for baner.\\
\begin{align*}
  O_a &= \{ x \in A \mid a \sim x \}\\
  &= \{ x \in A \mid \pi(g)a = x \text{ for et givent } g \in G \}\\
  &= \{ \pi(g)a \mid g \in G \}
\end{align*}

$$A = \bigcup_a O_a$$
Hvor alle baner er disjunkte.

\subsection*{Stabilisator}
\textbf{Definition.}\\
$$Stab_G(x) = \{ g \in G \mid \pi(G)x = x \}$$
Stabilisator undergruppen af x i G.\\
Denne undergruppe er ikke nødvendigvis normal.

Vi ved fra Lagrange, at følgende gælder:\\
$$|O_x| = [G : Stab_G(x)] = |G| / |Stab_G(x)|$$

\subsection*{Cauchys sætning}
Lad $G$ være en gruppe med endelig orden og lad $p$ være et primtal.\\
Hvis $p \mid |G|$, så $\exists{x \in G}$, sådan at $|x| = p$.

\section*{Cykliske grupper og undergrupper}
\subsection*{Definition}


\subsection*{Orden}

\section*{Kvotientgrupper}
\subsection*{Ækvivalens relationer}

\begin{enumerate}
  \item Refelksiv $x \sim x$.
  \item Symmetrisk $x \sim y \Rightarrow y \sim x$
  \item Transitiv $x \sim y, y \sim z \Rightarrow x \sim z$.
\end{enumerate}

En ækvivalens relation $\sim$ på $G$, giver en partition af $G$.\\
\begin{align*}
  E_x &= \{ y \in G \mid x \sim y\}\\
  \bigcup_x{E_x} &= G\\
  E_x &= E_y \text{ eller } E_x \cap E_y = \emptyset
\end{align*}

\subsection*{Kernen}

\subsection*{Normale undergrupper}
\textbf{Definition 2}:\\
En undergruppe $N$ af $G$ kaldes normal, og vi skriver $N \trianglelefteq G$, hvis vi for alle $g \in G$ har, at\\
$$gNg^{-1} := \{ gng^{-1} \mid n \in N \} = N.$$
\subsection*{Sideklasser}
\textbf{Definition 1}:\\
Lad $H \leq G$. For $g \in G$ defineres venstresideklassen af $H$ i $G$:\\
$$gH = \{ gh \mid h \in H \}.$$
\textbf{Proposition 1}:\\
For $u, v \in G$ haves $uH = vH$ hvis og kun hvis $v^{-1} u \in H$ hvis og kun hvis $u$ og $v$ begge er repræsentanter for samme venstresideklasse.\\
Mængden af venstresideklasser i $G$ udgør en partition af $G$:\\
$$G = \bigcup_{g \in G} gH.$$\\
og for vilkårlige $u, v \in G$ gælder enten $uH \cap vH = \emptyset$ eller $uH = vH$.
\subsection*{Kvotientgrupper}
\textbf{Sætning 1}:\\
Lad $N \trianglelefteq G$ og betragt mængden af venstresideklasser af $N$ i $G$. Denne mængde betegner vi $G / N$ (læses: "G module N"). På $G / N$ giver fastsættelsen
$$uN \cdot vN := (uv)N$$
for $u, v \in G$ en veldefineret komposition, og $G / N$ er en gruppe med denne komposition. Det neutrale element i $G / N$ er sideklassen $N$, og det inverse element til en sideklasse $gN$ er sideklassen $g^{-1} N$. $G / N$ med denne komposition kaldes kvotientgruppen eller gaktorgruppen af $G$ modulo $N$.\\

\subsection*{Noget med konjugation}

\subsection*{Lagrange}

\subsection*{Isomorfisætninger}

\section*{Direkte produkter og abelske grupper}
\subsection*{Def}
\subsection*{Struktursætningen for endeligt frembragte abelske grupper}

Lad være $G$ en endelig abelsk gruppe.\\
$G \cong Z_{n_1} \times \hdots \times Z_{n_k}$, hvor $n_1 \geq 2, n_i \mid n_{i+1}$.
\textbf{Korollar.}\\
Lad $G$ være en endelig abelsk gruppe. Isomorfi-klassen af $G$ er givet ud fra dens liste af cykliske grupper.\\
Altså vil to endelige abelske grupper kun være isomorfe, hvis de er frembragt af de samme cykliske grupper.

\subsection*{Elementardivisorer}

\subsection*{Invariante faktorer}

\section*{Gruppevirkninger}
\subsection*{Permutationsrepresentation}
\subsection*{(Kerner og stabilisatorer)}
\subsection*{Tro virkninger}
\subsection*{Cayley}
\subsection*{Noget med konjugation}
\subsection*{Klasseligning}
\subsection*{Sylows sætninger}
\textbf{Sætning 1.}\\
Lad $p$ være en primfactor med multiplicitet $n$ af ordenen af en endelig gruppe $G$. Så vil der eksistere en sylow p-undergruppe med orden $p^n$.

\textbf{Sætning 2.}\\
Givet en gruppe $G$ og et primtal $p$, vil alle sylow $p$-undergruppe vil være congruente. Dvs. hvis $H$ og $K$ er sylow $p$-undergrupper af $G$, så vil $\exists{g \in G}$ så $gHg^{-1} = K$.

\textbf{Sætning 3.}\\
Lad $p$ være en primfactor med multiplicitet $n$ af ordenen af en endelig gruppe $G$, så $|G| = p^n m$, hvor $n \geq 0, p \not{\mid} m$. Lad $n_p$ være antallet af sylow $p$-undergrupper af $G$.
\begin{enumerate}
  \item $n_p \mid m$
  \item $n_p \equiv 1 (\text{mod } p)$
  \item $n_p = [G : N_G(P)]$
\end{enumerate}

\end{document}
