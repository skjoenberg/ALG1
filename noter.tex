\documentclass[11pt]{article}
\usepackage[a4paper, hmargin={2.8cm, 2.8cm}, vmargin={2.5cm, 2.5cm}]{geometry}
\usepackage{eso-pic} % \AddToShipoutPicture
\usepackage{graphicx}
\usepackage[utf8]{inputenc}
\usepackage[T1]{fontenc}
\usepackage{stmaryrd}
\usepackage{natbib}
\usepackage{float}
\usepackage{amsmath}
\usepackage{dsfont}
\usepackage{pgf}
\usepackage{tikz}
\usepackage{fancyhdr}
\usepackage{lastpage}
\usepackage{booktabs}
\usepackage{lmodern}
\usepackage[normalem]{ulem}
\usetikzlibrary{arrows,automata}

\pagestyle{fancy}
\fancyhf{}
\setlength{\headheight}{2.5em}
\renewcommand{\headrulewidth}{0pt}
\fancyhead[R]{Jacob Evald}
\fancyfoot[C]{page \thepage\ of \pageref{LastPage}}

\begin{document}

\section*{Grupper}
\subsection*{Definition}
\begin{enumerate}
  \item Lukket under multiplikation.
  \item Assosiativ.
  \item Unikt identitetselement.
  \item Lukket under inversdannelse og hver invers er unik.
\end{enumerate}

En gruppe er abelsk, hvis alle elementer kommuterer.
\subsection*{Orden}
Ordenen af en gruppe $G$ er givet ved antallet af elementer i gruppen.\\
Ordenen af et element $x \in G$ er givet ved det laveste heltal $n$, så $x^n = 1$ (hvor $1$ er identitetselementet).\\
Et element har uendelig orden, hvis der ikke findes et $n$, som angivet ovenfor.\\
Et element har orden $1$, hvis og kun hvis det er identitetselementet.

\subsection*{Cancellation laws}
\begin{align*}
  au &= ua, \text{ så gælder } u = v\\
  ub &= vb, \text{ så gælder } u = v
\end{align*}
\subsection*{Generatorer}
Generatoren for en gruppe er en delmængde, givet ved, at alle elementer i gruppen kan blive udtrykt ved en endelig kombination (under gruppeoperationen) af elementerne (og deres inverse) i delmængden.
\subsection*{Symmetriske grupper}
Lad $\Omega$ være en ikke-tom mængde og lad $S_{\Omega}$ være en mængde af alle bijektioner fra $\Omega$ til sig selv. $S_{\Omega}$ kaldes den symmetriske gruppe af mængden $\Omega$. $S_{\Omega}$ indeholder alle permutationer af $\Omega$.\\
I specialtilfældet $\Omega = \{ 1,2,3, \hdots, n \}$ skrives den symmetriske gruppe $S_n$.\\
Ordenen af $|S_n| = n!$.
\subsection*{Diedergruppen af orden 2n}
Diedergruppen af orden $2n$ skrives $D_{2n}$.\\
For alle $n \geq 3$, lad $D_{2n}$ være sættet af symmetrier af en $n$-gon.\\
Der findes to basale operationer på diedergruppen. Henholdsvis spejlinger $s$ og rotationer $r$.\\
Alle elementer i diedergruppen kan beskrives ved $s^k r^i$, hvor $k = 0$ eller $1$ og $0 \leq i \leq n-1$.\\
Ordenen af diedergruppen er $|D_{2n}| = 2n$.\\
$r s = s r^{-1}$.\\
$r^i s = s r^{-i}$.\\
Diedergruppen kan også beskrives ved generatoren: $D_{2n} = \langle r, s | r^n = s^2 = 1, rs = s r^{-1} \rangle$.
\subsection*{Matrixgrupper}
\subsection*{Den alternerende gruppe}
\subsection*{Kvaterniongruppen Q8}
\subsection*{Kleins 4-gruppe}
2.5
\subsection*{Homomorphier}
\subsection*{Isomorphier}
\subsection*{Gruppevirkninger}


\section*{Undergrupper}
\subsection*{Definition}
\subsection*{Undergruppekriteriet}
\subsection*{Centralisator}
\subsection*{Normalisator}
\subsection*{Stabilisator}
\subsection*{Kernen}
\subsection*{Centrum}
\subsection*{Stabilisatorer og kerner af gruppevirkninger}
\subsection*{Undergruppe frembragt af en delmængde}

\section*{Cykliske grupper og undergrupper}
\subsection*{Def}
\subsection*{Orden}

\section*{Kvotientgrupper}
\subsection*{Kernen}
\subsection*{Normale undergrupper}
\textbf{Definition 2}:\\
En undergruppe $N$ af $G$ kaldes normal, og vi skriver $N \trianglelefteq G$, hvis vi for alle $g \in G$ har, at\\
$$gNg^{-1} := \{ gng^{-1} | n \in N \} = N.$$
\subsection*{Sideklasser}
\textbf{Definition 1}:\\
Lad $H \leq G$. For $g \in G$ defineres venstresideklassen af $H$ i $G$:\\
$$gH = \{ gh | h \in H \}.$$
\textbf{Proposition 1}:\\
For $u, v \in G$ haves $uH = vH$ hvis og kun hvis $v^{-1} u \in H$ hvis og kun hvis $u$ og $v$ begge er repræsentanter for samme venstresideklasse.\\
Mængden af venstresideklasser i $G$ udgør en partition af $G$:\\
$$G = \bigcup_{g \in G} gH.$$\\
og for vilkårlige $u, v \in G$ gælder enten $uH \cap vH = \emptyset$ eller $uH = vH$.
\subsection*{Kvotientgrupper}
\textbf{Sætning 1}:\\
Lad $N \trianglelefteq G$ og betraft mængden af venstresideklasser af $N$ i $G$. Denne mængde betegner vi $G / N$ (læses: "G module N"). På $G / N$ giver fastsættelsen
$$uN \cdot vN := (uv)N$$
for $u, v \in G$ en veldefineret komposition, og $G / N$ er en gruppe med denne komposition. Det neutrale element i $G / N$ er sideklassen $N$, og det inverse element til en sideklasse $gN$ er sideklassen $g^{-1} N$. $G / N$ med denne komposition kaldes kvotientgruppen eller gaktorgruppen af $G$ modulo $N$.\\

\subsection*{Noget med konjugation}
\subsection*{Lagrange}
\subsection*{Isomorfisætninger}

\section*{Direkte produkter og abelske grupper}
\subsection*{Def}
\subsection*{Endeligt frembragte abelske grupper}
\subsection*{Elementardivisorer}
\subsection*{Invariante faktorer}

\section*{Gruppevirkninger}
\subsection*{Permutationsrepresentation}
\subsection*{(Kerner og stabilisatorer)}
\subsection*{Tro virkninger}
\subsection*{Cayley}
\subsection*{Noget med konjugation}
\subsection*{Klasseligning}
\subsection*{Sylows sætninger}


\end{document}
